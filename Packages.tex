\usepackage[table]{xcolor}	 % TCM Allows coloring of tables
\usepackage{adjustbox}       % TCM Extends the graphicx package to do trimming and other adjustments.
\usepackage{algorithm}       % TCM Algorithm block See http://algorithms.berlios.de/
\usepackage{algorithmic}     % TCM Algorithm & Peudocode See http://en.wikibooks.org/wiki/LaTeX/Algorithms_and_Pseudocode
\usepackage{amsmath}         % Need for subequations
%\usepackage{amssymb}        % amssymb-telda and math symbols
\usepackage{booktabs}        % TCM Allows fancy Tables
\usepackage{boxedminipage}   % Boxes around figures
\usepackage{breqn}           % TCM Automatically breaks long equation lines - usually
\usepackage{changepage} 	 % TCM Allows one to temporarily change right and left margins
\usepackage{cite} 	 		 % TCM Allows improved handling of numeric citations including automatic ordering and ranges of multiple cites
\usepackage{enumitem} 		 % TCM Allows one to easily change the labels for enumerate list
%\usepackage{epsf}           % TCM This EPS file package is greatly depreciated for the graphicx package bundle. Use at own risk.
\usepackage{fancyhdr}
\usepackage{flafter}         % Floats should always appear after their definition
\usepackage[flushleft]{threeparttable}  % TCM This package allows for notes under tables.  Flushleft option flushes the note left margin of table (center is default)
\usepackage{geometry}		 % TCM Allows to customize paper size and margins.
\usepackage{graphicx}        % TCM Extends graphics package for figures. Provides op­tional ar­gu­ments to the \in­clude­graph­ics com­mand
\usepackage{epstopdf}        % TCM Converts eps images to PDF to use in graphicx package.  Must be loaded after graphicx package
\usepackage{longtable}       % TCM Allows for Automatic page breaks for long tables
%\usepackage{jneurosci}       % TCM The jneurosci bibliography style makes use of some commands - for example, \citeauthoryear.  Must include if using named or acmsmall bib style
\usepackage{ltablex}		 % TCM Modifies Tabularx Package by combining properties of tabularx and longtable.
\usepackage[maxfloats=36]{morefloats}      % Latex will handle more floats 18-36
\usepackage{moresize}        % TCM Adds \HUGE and \ssmall font sizes
\usepackage{multirow}        % TCM Allows multirow cells in tables (similar to merge and center in Excel)
%\usepackage{named}           % TCM Bibliography style that allows for more fine-tuned citing. Commands include \cite, \citeauthor, \citeyear and \shortcite (for after you've already used \citeauthor).
\usepackage{outlines}		 % TCM Needed for outlines
\usepackage{rotate}			 % TCM Performs rotations of floating environments (images w/ cations, etc)
\usepackage{rotating}        % Need for sidewaystable
\usepackage{scrextend}		 % TCM package that allows for KOMA-Script classes available for other classes: e.g., labeling lists etc.  See package documentation for further information.
\usepackage{pdflscape}		 % TCM Allows Landscape Page
%%\usepackage{qtree}         % Need for trees
\usepackage{setspace}        % TCM Allows for more intuitive commands for single and double spacing
\usepackage{siunitx}		 % TCM package required to ensure units are typeset properly with numbers e.g. \SI{10}{\kg\m\per\square\s}
\sisetup{
	output-exponent-marker=\ensuremath{
		\mathrm{E}
	}
}  % TCM Option allows for Scientific notation of Numbers.  Format is \num{#.####E##}
% \usepackage[superscript]{cite}		  % TCM Makes citations superscript instead of [#]
\usepackage{tabularx}		 % TCM Package which allows paragraphs in Tables
% \usepackage{thumbpdf}		  % TCM Comment out as causes warnings, can easily create thumbs in Adobe if needed.
\usepackage[colorinlistoftodos,textwidth=25mm,shadow,backgroundcolor=green]{todonotes}				  % TCM Allows margin comments
\usepackage{marginnote}				  % TCM Require for Todonotes in Align Envionment
\makeatletter
\renewcommand{\@todonotes@drawMarginNoteWithLine}{%
	\begin{tikzpicture}[remember picture, overlay, baseline=-0.75ex]%
     		\node [coordinate] (inText) {};%
     	\end{tikzpicture}%
     	\marginnote[{% Draw note in left margin
		\@todonotes@drawMarginNote%
      		\@todonotes@drawLineToLeftMargin%
	}]{% Draw note in right margin
		\@todonotes@drawMarginNote%
         	\@todonotes@drawLineToRightMargin%
     	}%
}
\makeatother
\newcommand{\slstodo}[2][]{
	\todo[caption={#2}, size=\small, #1]{
		\renewcommand{\baselinestretch}{0.5}
		\selectfont#2
		\par
	}
}
\usepackage{verbatim}        % TCM Verbatim text block
\usepackage{wrapfig}         % Wrap figures
\usepackage{xspace}          % Allows dynamic space in global text variables. This can allow you to just use \newCommandName rather than \newCommandName{}.
\usepackage[bookmarks=true,backref=page, hyperfigures=true, pdfpagelabels=false]{hyperref}
\hypersetup{
	bookmarksopen=true,				  % True: Open bookmark tree
	bookmarksnumbered=true, 		  % True: put section numbers in bookmarks
	breaklinks=true,				  % True: split the url over multiple lines
%     draft=true,						  % True: do not do any hyper linking
	filecolor=magenta, 				  % Color of file links
	citecolor=blue, 				  % Color of links to bibliography
	colorlinks=true, 				  % False: boxed links; true: colored links
	linkcolor=blue, 				  % Color of internal links
	linktocpage=true, 				  % True: Makes the page number of TOC the link vs the text
%     pagebackref=true,				  % True: Links references back to referring page
	pdfnewwindow=true,         		  % True: URL links in new window
	plainpages=false, 				  % True: do page number anchors as plain Arabic
	pdffitwindow=false,        		  % Window fit to page when opened
	pdfmenubar=true,           		  % Show Acrobat menu
	pdfstartview={FitH},       		  % Fits the width of the page to the window
	pdftoolbar=true, 				  % Show Acrobat toolbar
	pdfauthor={Author}, 			  % Author in PDF document properties
	pdfcreator={Author}, 			  % Creator of the document in PDF document properties
	pdfkeywords={Keyword1} {Keyword2} {Keyword3} {Keyword4} {Author}, % List of keywords in PDF document properties
	pdfproducer={Producer}, 		  % Producer of the document in PDF document properties
	pdfsubject={Subject},   % Subject of the document in PDF document properties
	pdftitle={Title of Dissertation}, % Title in PDF document properties
	unicode=false,             		  % Non-Latin characters in Acrobat bookmarks
	urlcolor=cyan 					  % Color of external links
}
\usepackage{bookmark}		 		  % TCM Eliminates Warning Bookmark level greater than one.  Must be loaded after Hyperref
\usepackage{acronym}