\chapter{INTRODUCTION} \label{ch:introduction}% Must have a blank line after every section label

About $1/3$ of the cerebral cortex is for visual processing in our brain \cite{grady1993vision}, so humans are basically vision-driven species.  That's why ``What it looks like?" may be the most intuitive question pop out when humans try to know about a new thing. When people encountered a new problem, we usually try to find out what the problem looks like by modeling it then we can analyze the problem and solve it. There exists multiple ways of modeling a problem and this dissertation discusses modeling graph partitioning problems via visualization. Visualization answers the exact question: ``What it looks like?".

This dissertation states three problems of three different areas: i) Partitioning \acp{RGG} into several bipartite backbone grids; II) Hierarchical Maximum Concurrent Flow Problems; iii) Precise and Concise Graphical Partitioning of Natural Numbers. These are all graph partitioning problems in three different areas. We use visualization to model the problem and developed new algorithms to solve them. Yet the techniques discussed here apply to a broader area of problems: computer networks, social networks, arithmetic, computer graphics and software engineering. Common terminologies accepted across the three areas have been used in this dissertation to guarantee people from all areas can understand concepts mentioned here.

Algorithms are a fascinating use case for visualization. To visualize an algorithm, we don’t merely fit data to a chart; there is no primary dataset. Instead there are logical rules that describe behavior. This may be why algorithm visualizations are so unusual, as designers experiment with novel forms to better communicate. This is reason enough to study them.

But algorithms are also a reminder that visualization is more than a tool for finding patterns in data. Visualization leverages the human visual system to augment human intellect: we can use it to better understand these important abstract processes, and perhaps other things, too.

Our research method builds the visual model of existing problem, adds available features to the model, then discovers new features by implementing existing algorithms and develops new algorithms to get close to our purpose and visualize the problem again. Finally, expand the problem to introduce new challenges. The whole procedure could be termed as Visualized Algorithm Engineering.